\documentclass{beamer}
\usepackage[utf8]{inputenc}
\usepackage[T1]{fontenc}
\usepackage[catalan]{babel}
\usepackage{amsmath,amsthm,amssymb}
\usepackage{lmodern}
\usepackage{microtype}
\usepackage{tikz-cd}
\usetikzlibrary{arrows}
\usepackage{minted}
\setminted{xleftmargin=1cm}

\usetheme{Boadilla}

\newcommand{\cat}[1]{\mathcal{#1}}
\newcommand{\id}{\mathrm{id}}

\DeclareMathOperator{\Obj}{Obj}

\beamertemplatenavigationsymbolsempty

\title{Monads in Haskell}
\author{Claudi Lleyda Moltó}
\institute[UAB]{Universitat Autònoma de Barcelona}
\date{}

\begin{document}
\begin{frame}
    \titlepage
\end{frame}

\section{Introduction}

\begin{frame}{Motivation}
    Functions in computer programming are not mathematical functions.
    \note<1>{They do all sort of weird things, like
        \begin{itemize}
            \item Return different outputs for the same inputs (not well
                defined)
            \item Modify global state (very foreign to mathematics)
        \end{itemize}
    }

    \pause

    \begin{itemize}
        \item reading from memory
        \item writing to memory
    \end{itemize}
    \note<2>{These can be all expressed in terms of memory IO, outside their
    inputs}

    \pause

    If they were, we could better reason about code and get benefits
    \note<3>{We is the compiler\par}
    \begin{itemize}
        \item memorization
        \item automatic parallelization
    \end{itemize}
    \note<3>{Explain memorization and parallelization\par}
    \note<3>{other benefits include code safety
        \begin{itemize}
            \item not letting your functions interfere with others
            \item not letting your functions be affected by the effects of
                others
        \end{itemize}
    }
\end{frame}

\begin{frame}{Definitions}
    \note{Definitions from the original report\par}
    \begin{definition}[Procedure]
        We call functions in computer programming~\emph{procedures}.
    \end{definition}
    \note<1>{We may call simply them functions if the context is clear}

    \pause

    \begin{definition}[Pure function]
        If a procedure reads from memory we say it is~\emph{non-deterministic},
        and if it writes to memory we say it has~\emph{side-effects}.

        We say that a procedure that is deterministic and does~\alert{not} have
        side-effects is a~\emph{pure} function.
    \end{definition}
    \note<2>{Pure functions behave like mathematical functions, and we can treat
    them like such when reasoning}
\end{frame}

\section{Kleisli triples}

\begin{frame}{Procedure purifying examples}
    We can make an impure function pure by changing its codomain.

    \note{Impure functions in Set can be viewed as pure in an appropriate
    category\par}

    \pause

    \begin{example}[Non-determinism]
        If~\(\tilde{f}:A\longrightarrow B\) is non-deterministic, we can define
        a function of all possible outputs
        \[
            f:A\longrightarrow\mathcal{P}(B)
        \]
        which becomes deterministic
        \note<2>{We can use the power set as codomain to encapsulate all
        possibilities}
    \end{example}

    \pause

    \begin{example}[Side-effects]
        If~\(\tilde{f}:A\longrightarrow B\) has side-effects, we can define a
        function to track the effects on the set of states~\(S\) and the
        original result
        \[
            f:S\times A\longrightarrow S\times B
        \]
        which has no side-effects.
        \note<3>{Explain exponential notation}
    \end{example}
\end{frame}

\begin{frame}{Procedure purifying examples}
    We can make an impure function pure by changing its codomain.

    \begin{example}[Non-determinism]
        If~\(\tilde{f}:A\longrightarrow B\) is non-deterministic, we can define
        a function of all possible outputs
        \[
            f:A\longrightarrow\mathcal{P}(B)
        \]
        which becomes deterministic
    \end{example}

    \begin{example}[Side-effects]
        If~\(\tilde{f}:A\longrightarrow B\) has side-effects, we can define a
        function to track the effects on the set of states~\(S\) and the
        original result
        \[
            f:A\longrightarrow(S\times B)^{S}
        \]
        which has no side-effects.
    \end{example}
\end{frame}

\begin{frame}{Procedure purifying abstraction}
    \begin{definition}[Kleisli triples]
        We define a~\emph{Kleisli triple} over a category~\(\cat{C}\) as a
        triple~\((T,\eta,(-)^{\ast})\) consisting of
        \begin{itemize}
            \item a class function
                \(T:\Obj(\cat{C})\longrightarrow\Obj(\cat{C})\).

            \item for all~\(A\in\cat{C}\)
                a morphism~\(\eta_{A}:A\longrightarrow TA\)
                in~\(\cat{C}\).

            \item for all~\(f:A\longrightarrow TB\) a
                morphism~\(f^{\ast}:TA\longrightarrow TB\).
        \end{itemize}
        such that
        \begin{itemize}
            \item for all~\(A\in\cat{C}\) we have
                \({\eta_{A}}^{\ast} = \id_{TA}:TA\longrightarrow TA\).

            \item for all~\(A,B\in\cat{C}\) and \(f:A\longrightarrow TB\) we
                have~\(f^{\ast}\circ\eta_{A} = f\).

            \item for all~\(f:A\to TB,g:B\to TC\) we
                have~\(g^{\ast}\circ f^{\ast} = (g^{\ast}\circ f)^{\ast}\).
        \end{itemize}
    \end{definition}
    \note{This is a very powerful definition, as shown in the next slide}
\end{frame}

\begin{frame}[fragile]{Kleisli triples and monads}
    \begin{theorem}[Monad]
        A Kleisli triple~\((T,\eta,(-)^{\ast})\) induces
        \begin{itemize}
            \item an endofunctor~\(T:\cat{C}\rightarrow\cat{C}\),
                by setting~\(f\mapsto(\eta_{B}\circ f)^{\ast}\)

            \item two natural transformations~\(\eta:\id\rightarrow T\)
                and~\(\mu:TT\rightarrow T\) with commutative diagrams
                \[
                    \begin{tikzcd}
                        TA \arrow{r}{T\eta_{A}} \arrow{dr}{} & TTA \arrow{d}{\mu_{A}} & TA \arrow{l}[swap]{\eta_{TA}} \arrow{dl}{} \\
                                                             & TA \arrow{ul}{\id_{TA}} \arrow{ur}[swap]{\id_{TA}} &
                    \end{tikzcd}
                    \qquad\text{and}\qquad
                    \begin{tikzcd}
                        TTTA \arrow{r}{\mu_{TA}} \arrow{d}{T\mu_{A}} & TTA \arrow{d}{\mu_{A}} \\
                        TTA \arrow{r}{\mu_{A}} & TA
                    \end{tikzcd}
                \]
                by setting~\(\mu=(\id_{T})^{\ast}\).
        \end{itemize}
    \end{theorem}

    \note<1>{
        \begin{itemize}
            \item The object part of~\(T\) and the natural
                transformation~\(\eta\) are the ones defined by the Kleisli
                triple.

            \item This precisely is the definition of
                a~\emph{monad}~\((T,\eta,\mu)\).

            \item Every Kleisli triple induces a monad.

            \item In fact, we will see that the converse is also true.
        \end{itemize}
    }

    \pause

    \begin{theorem}
        A monad~\((T,\eta,\mu)\) induces a Kleisli: triple for any
        morphism~\(f:A\rightarrow B\) set~\(f^{\ast}=\mu_{B}\circ Tf\).
    \end{theorem}

    \note<2>{
        \begin{itemize}
            \item The maps~\(T\) and~\(\eta\) are the ones induced by the
                Kleisli triple.

            \item Monads are important in computer science because lots of
                commonly use structures in computer science are monads, so
                morally we are not increasing the complexity of the field by
                introducing the notion of Kleisli triples, as they are the same
                thing as monads.

            \item Monads are also preferred because they are more prevalent in
                research in Category Theory, and are defined in terms of natural
                transformations and functors, making them more suitable for
                abstract manipulation.
        \end{itemize}
    }
\end{frame}

\section{Haskell}

\begin{frame}{Haskell}
    \begin{block}{Remark}
        \begin{itemize}
            \item We have an abstract solution for a computer science problem.

                \note<1>{So far, we have introduced an abstraction that should
                    allow us to have a programming language where all functions are
                pure, such that the compiler can reason about them.\par}

            \item It might not be useful in practice.

                \note<1>{It is not clear how it would be implemented or if it
                would be feasible.\par}
        \end{itemize}
    \end{block}

    \pause

    Haskell, introduced in 1990, is a general-purpose programming language where
    all functions are pure, and does so by successfully implementing these
    abstractions.

    \note<2>{We cannot introduce a full blown programming language in a 15
    minutes presentation, but we can explore some of its features}
\end{frame}

\begin{frame}[fragile]{Haskell}
    \note<1>{Broadly speaking\par}
    \note<1>{Define \emph{type}}

    A~\mintinline{Haskell}{Monad m} in Haskell is a~\emph{type} constructor, and
    has two operations
    \begin{minted}{Haskell}
(>>=) :: m a -> (a -> m b) -> m b
    \end{minted}
    and
    \begin{minted}{Haskell}
return :: a -> m a
    \end{minted}

    \note<1>{\begin{itemize}
            \item The first operation is called~\emph{bind}

            \item Read lines out loud

            \item Mention multiple arguments
    \end{itemize}}

    \pause

    which correlate as

    \begin{align*}
        \text{\mintinline{Haskell}{m}} &\Longleftrightarrow T \\
        \text{\mintinline{Haskell}{>>=}} &\Longleftrightarrow (-)^{\ast} \\
        \text{\mintinline{Haskell}{return}} &\Longleftrightarrow \eta
    \end{align*}

    \note<2>{Start with~\mintinline{Haskell}{return}

        The trick for bind is to switch the argument order

        Combining these functions with~\emph{strong typing} and some other
    programming features is enough.}
\end{frame}

\begin{frame}[fragile]{Haskell}
    A~\mintinline{Haskell}{Monad m} in Haskell is a~\emph{type} constructor, and
    has two operations
    \begin{minted}{Haskell}
(>>=) :: (m a, (a -> m b)) -> m b
    \end{minted}
    and
    \begin{minted}{Haskell}
return :: a -> m a
    \end{minted}
    which correlate as

    \begin{align*}
        \text{\mintinline{Haskell}{m}} &\Longleftrightarrow T \\
        \text{\mintinline{Haskell}{>>=}} &\Longleftrightarrow (-)^{\ast} \\
        \text{\mintinline{Haskell}{return}} &\Longleftrightarrow \eta
    \end{align*}
\end{frame}

\begin{frame}[fragile]{Haskell}
    A~\mintinline{Haskell}{Monad m} in Haskell is a~\emph{type} constructor, and
    has two operations
    \begin{minted}{Haskell}
(>>=) :: ((a -> m b), m a) -> m b
    \end{minted}
    and
    \begin{minted}{Haskell}
return :: a -> m a
    \end{minted}
    which correlate as

    \begin{align*}
        \text{\mintinline{Haskell}{m}} &\Longleftrightarrow T \\
        \text{\mintinline{Haskell}{>>=}} &\Longleftrightarrow (-)^{\ast} \\
        \text{\mintinline{Haskell}{return}} &\Longleftrightarrow \eta
    \end{align*}
\end{frame}

\begin{frame}[fragile]{Haskell}
    A~\mintinline{Haskell}{Monad m} in Haskell is a~\emph{type} constructor, and
    has two operations
    \begin{minted}{Haskell}
(>>=) :: (a -> m b) -> (m a -> m b)
    \end{minted}
    and
    \begin{minted}{Haskell}
return :: a -> m a
    \end{minted}
    which correlate as

    \begin{align*}
        \text{\mintinline{Haskell}{m}} &\Longleftrightarrow T \\
        \text{\mintinline{Haskell}{>>=}} &\Longleftrightarrow (-)^{\ast} \\
        \text{\mintinline{Haskell}{return}} &\Longleftrightarrow \eta
    \end{align*}
\end{frame}

\begin{frame}{Conclusions}
    \begin{itemize}
        \item The abstraction can be applied rather directly, and it
            successfully leads to the previously mentioned benefits.

            \pause

        \item This is done by expanding on concepts already found in computer
            science.

            \pause

        \item These features have been introduced into most other languages to
            some extent.
    \end{itemize}
\end{frame}

\begin{frame}
    \begin{center}
        \Huge Thanks!
    \end{center}
\end{frame}
\end{document}
